\documentclass[a4paper,12pt,leqno]{article}
\usepackage[utf8]{inputenc}
\usepackage[T1]{fontenc}
\usepackage[polish]{babel}
\usepackage{amsmath}
\usepackage{a4wide}
\usepackage{graphicx}
\usepackage{subfig}
\usepackage{wrapfig}

\newcommand{\nel}[1]{|#1|}
\usepackage{program}

\title{\textbf{Algorytmy ewolucyjne}\\
       {\Large Raport z zadania pierwszego}\\[-1ex]}
\author{Karol Konaszyński i Wiktor Janas}
\date{Wrocław, dnia \today\ r.}

\newcommand{\avg}{\mathrm{avg}}
\newcommand{\dist}{\mathrm{dist}}
\newcommand{\median}{\mathrm{median}}
\newcommand{\target}{\mathrm{target}}
\newcommand{\RETURN}{|return|\ }

\begin{document}
\maketitle

Tematem tego projektu jest zastosowanie algorytmów ewolucyjnych do rozpoznawania obrazów. 
Problem rozpoznawania obrazów ma liczne zastosowania: od analizy zdjęć satelitarnych poprzez monitorowanie ruchu drogowego, po przemysł filmowy.
Z drugiej rozwiązanie go okazuje się być trudne, a pojęcie ,,podobieństwa obrazów'', oczywiste dla człowieka, nie jest precyzyjnie zdefiniowane.
Podejście zaproponowane w tej pracy polega na analizie samego kształtu obrazu, który uzyskujemy poprzez analizę jasności poszczególnych
pikseli (z pominięciem informacji o kolorze).

Przyjęliśmy następujące założenia: dana jest baza obrazów o zadanych nazwach (patrz rys. \ref{knownpics}). Każdy obraz, który chcemy rozpoznać
(zapytanie) jest przyrównywane dp wszystkich obrazów z bazy. Spośród obrazów bazowych wybieramy te, których odległość do zapytania jest najmniejsza.

\section{Znajdowanie punktów charakterystycznych}
Pierwszym etapem przetwarzania obrazu (pochodzącego z bazy danych lub będącego zapytaniem) jest znalezienie na nim punktów charakterystycznych.
Niech $\avg(x,y,d)$ oznacza średnią jasność obrazu w kwadracie o środku w $(x,y)$ i boku długości $d$. Dla piksela o współrzędnych $(x_0,y_0)$ oraz danej
wartości $d$ można zdefiniować funkcję
\[ F_d(\alpha) = \avg(x_0,y_0,d) - \avg( x_0+d\sin\alpha, y_0+d\cos\alpha, d) \]
Intuicyjnie jest to gradient piksela $(x_0,y_0)$ w kierunku $\alpha$ na obrazie zmniejszonym $d$-krotnie. Przyjmując kilka ustalonych wartości $d$
(w naszej implementacji były to $1,3,8$), można zdefiniować funkcję
\[ G(\alpha) = \sqrt[3]{F_1(\alpha) \cdot F_3(\alpha) \cdot F_8(\alpha)} \]
Następnie można obliczyć amplitudę tej funkcji:
\[ A = \max_\alpha G(\alpha) - \min_\alpha G(\alpha) \]
Wartość tą przyjmujemy za ,,miarę ciekawości'' piksela. 

Intuicja stojąca za tym algorytmem jest prosta -- szukamy punktów, w otoczeniu których następuje zmiana jasności obrazu. Miarą zmiany jasności w danym
kierunku jest gradient, zatem funkcja $F$ opisuje zmianę jasności obrazu we wszystkich kierunkach. Musimy jednak wziąć pod uwagę, że niektóre zmiany są
jedynie lokalne. Aby wyeleminować szum, obliczamy średnią geometryczbą funkcję $F$ dla kilku skal (jest to funkcja $G$) -- jeżeli istotnie zachodzi
rozpatrywany piksel leży w obszarze silnych zmian, nastąpi rezonans, gdy natomiast zmiana jest lokalna, funkcja większej skali ją ,,wyciszy''.
Według naszej całkowicie subiektywnej oceny skale 1,3,8 spradzają się najlepiej.

Mając określoną miarę ciekawości każdego piksela, chcemy wybrać określoną liczbę punktów charakterystycznych, które będą opisywały kształt obiektu
(liczba ta jest parametrem alogorytmu, najczęściej przyjmowaliśmy 50). Do tego celu zastosowaliśmy następujący algorytm: w każdym kolejnym kroku wybieramy
najciekawszy piksel na obrazku i dodajemy go do zbioru punktów interesujących. Nastepnie pewne jego otoczenie oznaczamy jako tabu i powtarzamy procedurę tak
długo, aż wszystkie punkty zostaną tabu. Aby określić wielkość tabu, zastosowaliśmy heurystykę ,,im ciekawiej tym gęściej''. Mianowicie, wielkość tabu jest
odwrotnie proporcjonalna do miary ciekawości punktu z pewnym współczynnikiem, który jest dobierany (za pomocą wyszukiwania binarnego) tak, aby ostateczny
zbiór punktów żądaną określoną wielkość. Proces wyszukiwania ciekawych punktów ilustruje rysunek \ref{poisrch}.

\begin{figure}\centering
\subfloat[Jabłko]{\includegraphics[width=4.5cm,keepaspectratio=true]{./apple-mod-pois.png}}
\subfloat[Wisienki]{\includegraphics[width=4.5cm,keepaspectratio=true]{./cherries-pois.png}}
\subfloat[Trójliterówka]{\includegraphics[width=4.5cm,keepaspectratio=true]{./jeb-pois.png}}
\caption{Przykładowe obrazy z bazy danych wraz z punktami charakterystycznymi}\label{knownpics}

\subfloat[gradient, $d = 1$]{\includegraphics[width=5.75cm,keepaspectratio=true]{./noisy-grad-1.png}}\hspace{5mm}
\subfloat[gradient, $d = 3$]{\includegraphics[width=5.75cm,keepaspectratio=true]{./noisy-grad-3.png}}\\
\subfloat[gradient, $d = 8$]{\includegraphics[width=5.75cm,keepaspectratio=true]{./noisy-grad-8.png}}\hspace{5mm}
\subfloat[gradient, łącznie]{\includegraphics[width=5.75cm,keepaspectratio=true]{./noisy-grad.png}}\\
\subfloat[znalezione punkty]{\includegraphics[width=5.75cm,keepaspectratio=true]{./noisy-pois.png}}
\caption{Proces wyszukiwania punktów charakterystycznych}\label{poisrch}
\end{figure}

\section{Analiza podobieństwa obrazów}

Mając określone punkty charakterystyczne, możemy zająć się badaniem podobienstwa obrazków.
Odległość od zbioru punktów $\bar P$ do zbioru $\bar Q$ definiujemy jako
\[ \dist(\bar P \rightarrow \bar Q) = \frac{1}{\nel{\bar P}} \cdot \sum_{p \in \bar P} \min_{q \in \bar Q}(\|p - q\|) \]
Odległość między zbiorami $\bar P$ a $\bar Q$ definiujemy jako
\[ \dist(\bar P, \bar Q) = \frac{\nel{\bar P} \cdot \dist(\bar P \rightarrow \bar Q) + \nel{\bar Q} \cdot \dist(\bar Q \rightarrow \bar P)}{\nel{\bar P} + \nel{\bar Q}} \]
Za miarę podobieństwa dwóch zbiorów punktów przyjmujemy
\[ \mathrm{similar}(\bar P, \bar Q) = \inf\{\dist(\bar P, M\bar Q) \;\arrowvert\; M\text{: afiniczne}\} \]
Jednak aby znaleźć tę musimy znaleźć przekształcenie minimalizujące wartość $\dist$. I tutaj wkracza ewolucja. 

\section{Ewolucja}

Naszymi osobnikami będą macierze afiniczne, czyli macierze postaci
$\begin{pmatrix}
\star & \star & \star \\
\star & \star & \star \\
0 & 0 & 1
\end{pmatrix}$, gdzie drugi minor główny jest złożeniem obrotu i skalowania, natomiast ostatnia kolumna odpowiada za translację.
Funkcją celu jest zatem 
\[ \target = -\dist(\bar P,  M\bar Q) \]
gdzie $\bar P$ jest wektorem punktów charakterystycznych obrazka z zapytania, zaś $\bar Q$ -- wektorem tychże dla obrazka z bazy.
Celem ewolucji jest zmaksymalizowanie funkcji celu, czyli znalezienie przekształcenia minimalizującego odległość.

Do rozwiązania tego problemu wykorzystaliśmy ideę Differential Evolution. Pseudokod zaimplementowanego algorytmu przedstawiony został na listingu: 
\begin{program}
\FUNCT |evolve| \BODY
    P := \{\}
    \FOR i := 1 \TO N \DO
        P := P + |random_translation| * |random_rotation| * |random_scaling|
    \END
    \WHILE \NOT |termination_condition| \DO
        \FOR i := 1 \TO N \DO
            |evaluate|(P_i)
        \END
        |sort|(P) 
        ~
        P := \set{P_i | i \leq N*|replace\_rate| }
        \FOR i := 1 \TO N*|replace_rate| \DO
            \IF |random|(|de_prop|)
                \THEN P := P + |de_crossover|(P);
                \ELSE P := P + |roulette_selection|(P); \FI
        \END
        ~
        \FOR i := 1 \TO N \DO 
            \IF |random|(|t_prop|)
	        \THEN P_i := |random_translation| * P_i; \FI
            \IF |random|(|s_prop|)
                \THEN P_i := |random_scaling| * P_i; \FI
            \IF |random|(|r_prop|)
                \THEN P_i := |random_rotation| * P_i; \FI
	    \IF |random|(|f_prop|)
                \THEN P_i := |random_flip| * P_i; \FI
        \END
    \END
\END
~
\FUNCT |de_crossover|(P) \BODY
    x1 := |roulette_selection|(P)
    x2 := |roulette_selection|(P),\; x1 \neq x2
    x3 := |roulette_selection|(P),\; x3 \neq x1,\; x3 \geq x2
    \RETURN x1 + (x3-x2) * f_\text{de}
\END
~
\FUNCT |termination_condition| \BODY
    \RETURN |gen_idx| > G \OR 
	    |max_target|(|gen_idx|, |gen_idx|-K) \geq |max_target|(|gen_idx|-K, |gen_idx|-2*K)
\END
\end{program}

Najpierw odbywa się losowa inicjacja populacji z parametrem N. Następnie wchodzimy w pętlę ewolucji, która wykonuje się aż nie zostanie spełniony
warunek zakończenia. W pętli, pierwszą czynnością jest ewaluacja każdego osobnika, czyli znalezienie dla niego wartości target oraz fitness,
gdzie fitness jest współczynnikiem przystosowania:
\[ \mathrm{fitness}(p) = \frac{\target(p) - \min_{q \in P}\target(q)}{\sum_{r \in P} \target(r) - \min_{q \in P}\target(q)} \]
Wartość fitness jest wykorzystywana do uporządkowania osobników od najlepszego do najgorszego oraz podczas selekcji ruletkowej.
Potem następuje selekcja blokowa połączona z krzyżowaniem, ostatecznie zaś mutacja każdego osobnika. Warunkiem zakończenia ewolucji
jest przekroczenie pewnej ustalonej ilości pokoleń (zwykle rzędu 200) bądź stwierdzenie, że w czasie ostatnich $K$ pokoleń nie nastąpiła
poprawa w stosunku do poprzednich $K$ pokoleń ($K$ jest również parametrem).

Krzyżowanie polega na wybraniu metodą ruletki trzech różnych osobników $x_1, x_2, x_3$ a następnie utworzeniu z nich osobnika postaci
$x1 + f_\text{de}\cdot(x3-x2)$, gdzie $f_\text{de}$ jest parametrem, zaś operacja dodawania i mnożenia jest zwykłą operacją dodawania
macierzy i mnożenia ich przez skalar.

Ważną obserwacją którą poczyniliśmy jest to, że nie warto dokonywać losowych skalowań, czy obrotów wokół punktu $(0,0)$, gdyż to spowoduje duże
zmiany w położeniu punktów oddalonych od środka układu współrzędnych (czyli lewego górnego rogu). Aby tego uniknąć, obroty i skalowania wykonujemy
względem punktu wybranego z małego otoczenia punktu $(\median(\bar X), \median(\bar Y))$, gdzie $\bar X, \bar Y$ są zbiorami współrzędnych x, y
punktów charakterystycznych.

\section{Eksperymenty}
Przeprowadzane eksperymenty podzieliliśmy na kilka grup: rozpoznawanie owoców, cyfr, słów trzyliterowych oraz obrazków bardzo zaszumionych.
Przykładowe obrazki użyte do testów przedstawiamy na rysunku 2.
Opiszemy tutaj kilka z eksperymentów, a następnie przedstawimy wyniki kilku następnych.
Eksperyment polega na ustaleniu grupy obrazków bazowych (rzędu 10 - 20), wybraniu obrazka do zapytania, który jest "podobny" do któregoś z bazowych a następnie uruchomieniu algorytmu
i wypisaniu pięciu najbardziej podobnych obrazków bazowych wraz z ich odległościami. Na tej podstawie, już czysto optycznie, stwierdzimy czy program rzeczywiście dał "poprawne" odpowiedzi.
Przedstawimy tutaj zarówno eksperymenty udane jak również nieudane, aby wyciągnąć kilka wniosków na przyszłość.

\subsection{Parametry ewolucji użyte w testach}
Poniżej przedstawiam zestaw parametrów, które możemy modyfikować.

\begin{tabular}{l|l|l}
parametr                & wartość & znaczenie \\
\hline
\texttt{threads}        & 2     & ilość wątków, na których działa program \\
\hline
\texttt{poiSteps}       & 16    & skok funkcji gradientu od kąta \\
\texttt{poiScales}      & 1,3,8 & skale gradientów \\
\texttt{poiCount}       & 60    & ilość punktów charakterystycznych (POI) \\
\texttt{poiThreshold}   & 30    & próg poniżej którego nie uznajemy punktu za POI \\
\hline
\texttt{populationSize} & 400   & wielkość populacji \\
\texttt{survivalRate}   & .8    & współczynnik przetrwania \\
\texttt{stopCondParam}  & 40    & wartość K używana w warunku zakończenia \\
\texttt{maxGeneration}  & 200   & maksymalna ilość pokoleń \\
\hline
\texttt{translateInit}  & 0.5   & początkowy rozrzut translacji\\
\texttt{rotateInit}     & 6.283 & początkowy rozrzut rotacji\\
\texttt{scaleInit}      & 4     & początkowy rozrzut skali \\
\hline
\texttt{translateProp}  & .03   & prawdop. translacji podczas mutacji \\
\texttt{rotateProp}     & .03   & prawdop. obrotu podczas mutacji \\
\texttt{scaleProp}      & .03   & prawdop. skalowania podczas mutacji \\
\texttt{flipProp}       & .05   & prawdop. odbicia względem prostej podczas mutacji \\
\hline
\texttt{translateDev}   & .5    & odchylenie w losowaniu translacji (w mutacji) \\
\texttt{rotateDev}      & .05   & odchylenie w losowaniu obrotu (w mutacji) \\
\texttt{scaleDev}       & .05   & odchylenie w losowaniu skalowania (w mutacji) \\
\texttt{originDev}      & 0.3   & rozrzut przy wyborze punktu srodka obrotu i skali \\
\hline
\texttt{deMatingProp}   & 0.8   & prawdop. krzyżowania DE \\
\texttt{deMatingCoeff}  & 0.1   & współczynnik przy krzyżowaniu DE \\
\texttt{deMatingDev}    & 0.05  & rozrzut powyższego \\
\end{tabular}



\subsection{Szukanie jabłka wśród owoców}
Ten test jest prostym benchmarkiem, ale ważnym i dobrze ilustrującym co się dzieje w naszym algorytmie. Mamy bazę zdjęć owoców, konkretnie: 
pojedynczą wiśnię, dwie wisienki, banan, jabłko, ananas, porcja arbuza, papryka i gruszka. Zapytaniem jest identyczne jak w bazie jabłko, przy czym obrócone, pomniejszone i przesunięte.
Oczekiwanym przez nas wynikiem jest zwrócenie przez algorytm listy najbardziej podobnych owoców z jabłkiem na szczycie listy i odległością jak najbliższą zera.

Na Rysunku 3. przedstawione są funkcje celu najlepszych osobników w kolejnych pokoleniach. Istotnie, ostatecznie odległość obrazka z zapytania od jabłka z bazy jest niewielka, 
istotnie mniejsza niż od pozostałych owoców. Tak więc eksperyment uznajemy za udany, a jego wynik - zgodny z oczekiwaniami.

\subsection{Dopasowanie wisienek}
Ten test z kolei jest dosyć trudny. Dostajemy bowiem jako zapytanie dwie wisienki, bardzo podobne do wisienek z bazy, te jednak mają listek na ogonku, co powoduje,
że obrazki przestają być bardzo podobne. My jednak w dalszym ciągu chcemy, żeby wisienki były najbliższym mu obrazkiem. Oto wynik działania algorytmu:
\begin{verbatim}
cherries, fruits (-12.547963)
banana, fruits (-14.550430)
pineapple, fruits (-14.859610)
paprika, fruits (-14.931395)
watermelon, fruits (-15.112849)
\end{verbatim}
Udało się! Nic dziwnego, że odległość nie jest tak wyraźnie mniejsza od pozostałych niż było to w przypadku jabłka, jednak cel został osiągnięty. 
Znalezione dopasowanie przedstawione jest na Rysunku 4.

\subsection{Inne eksperymenty z owocami}

\paragraph{Gruszka}


Wzbudzającym 

pineapple, fruits (-11.685663)
watermelon, fruits (-12.371205)
apple, fruits (-12.665856)
paprika, fruits (-13.042226)
pear, fruits (-13.062563) (działa)


\begin{figure}\centering
\footnotesize\include{applemod-vs-fruits}\vspace{-2em}
\normalsize\caption{Dopasowanie owoców (\texttt{fruits}) do obróconego jabłka (\texttt{applemod}); uśrednione wyniki z dziesięciu uruchomień programu.}
\end{figure} 

\begin{figure}\centering
\includegraphics[width=6cm,keepaspectratio=true]{./cherries-match.png}
\caption{Dopasowanie \texttt{cherries} do \texttt{cherries6}}
\end{figure}

\begin{figure}\centering
\footnotesize\include{square-vs-romb-popsize}\vspace{-2em}
\normalsize\caption{Wpływ rozmiaru populacji na ewolucję (dopasowanie \texttt{square} do \texttt{romb}); uśrednione wyniki z dziesięciu uruchomień programu.}
\end{figure}
\begin{figure}\centering
\footnotesize\include{square-vs-romb-deprop}\vspace{-2em}
\normalsize\caption{Wpływ prawdopodobieństwa krzyżowania na ewolucję (dopasowanie \texttt{square} do \texttt{romb}); uśrednione wyniki dziesięciu uruchomień programu.}
\end{figure}
\begin{figure}\centering
\footnotesize\include{square-vs-romb-surv}\vspace{-2em}
\normalsize\caption{Wpływ współczynnika przetrwania na ewolucję (dopasowanie \texttt{square} do \texttt{romb}); uśrednione wyniki dziesięciu uruchomień programu.}
\end{figure}

\begin{figure}\centering
\includegraphics[width=6cm,keepaspectratio=true]{./cherries-match.png}
\caption{Dopasowanie \texttt{cherries} do \texttt{cherries6}}
\end{figure}

\section{Uwagi implementacyjne}

Program wymaga systemu operacyjnego GNU/Linux. Do jego kompilacji niezbędne są, poza standardowymi narzędziami, pliki nagłówkowe biblioteki GTK+,
zazwyczaj dostępne w pakiecie \texttt{libgtk2.0-dev} lub podobnym. Aby skompilować program należy w katalogu ze źródłami wydać polecenie \texttt{make}.
W wyniku jego wykonania powinien powstać plik wykonywalny o nazwie \texttt{evolution}. Program przyjmuje dwa parametry, plik z obrazkiem-zapytaniem oraz
nazwę grupy obrazków z bazy danych. Przykładowe wywołanie może zatem wyglądać następująco:
\begin{verbatim}./evolution fruits/apple-mod.pgm fruits\end{verbatim}
Powyższe polecenie powoduje porównanie obrazka \texttt{fruits/apple-mod.pgm} do wszystkich obrazków z kategorii \texttt{fruits}.
Kategorie są opisane w pliku \texttt{evolution.database} i można je edytować. Gdy program otrzymuje obraz do analizy po raz pierwszy, niezbędne jest
wyszukanie punktów charakterystycznych. Jest to długotrwała czynność (może trwać nawet pół minuty dla jednego obrazka), jednak znalezione punkty
charakterystyczne są zapisywane na dysk i przy kolejnych uruchomieniach program będzie korzystał z obliczonych wcześniej danych. Program wczytuje
obrazy w formacie \texttt{pgm}, wraz z kodem źródłowym dostarczane są przykładowe obrazy (w katalogach \texttt{digs}, \texttt{fruits}, \texttt{geom},
\texttt{noise} oraz \texttt{text}). Uwaga: program nie potrafi wczytać obrazów, które zawierają komentarze. Pliki \texttt{pgm} tworzone przez program
GIMP zawierają komentarz (i nie jest nam znana żadna metoda skłonienia GIMPa do nie umieszczania komentarza w plikach \texttt{pgm}). Wraz z kodem
źródłowym dostarczono skrypt \texttt{rmhash}, który usuwa komentarze z plików \texttt{pgm}.

Kod źródłowy podzielony jest na kilka plików. Główna część programu znajduje się w pliku \texttt{evolution.C}. Plik \texttt{poi.C} zawiera algorytm
wyszukiwania punktów charakterystycznych. Plik i \texttt{image.C} oraz \texttt{composite.C} zawierają kod operacji na obrazkach. Pliki \texttt{gui.C}
oraz \texttt{gui-gtk.c} zawierają implementację graficznego interfejsu użytkownika. Plik \texttt{workers.C} zawiera framework dla przetwarzania
wielowątkowego, zaś plik \texttt{config.C} zawiera parser konfiguracji.

\end{document}
